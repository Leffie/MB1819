%\documentclass[11pt,a4paper]{article}
\documentclass[a5paper,portuguese,twoside,10pt]{article}
\usepackage[a5paper]{geometry}

\parindent 0pt

\newcommand{\titulo}[1]{\Large{\textbf{#1}\vspace*{.2cm}} }
\newcommand{\autor}[2]{\normalsize{\textsf{#1 #2} \index{\textbf{#2}, #1}}}
\newcommand{\afil}[2]{\mbox{} \\ #1, \textit{#2} \vspace*{.1cm} \\}
\newcommand{\pchave}[1]{{\textbf{Palavras--chave}: #1\\}\vspace{-0.2cm}}

\newcommand\halmos{\rule{0.1in}{0.1in}}
\newtheorem{teorema}{Teorema}[section]
\newtheorem{lema}[teorema]{Lema}
\newtheorem{proposicao}[teorema]{Proposi\c{c}\~{a}o}
\newtheorem{corolario}[teorema]{Corol\'{a}rio}
\newtheorem{definicao}[teorema]{Defini\c{c}\~{a}o}
\newtheorem{exemplo}[teorema]{Exemplo}
\newtheorem{algoritmo}[teorema]{Algoritmo}
\newtheorem{observacao}[teorema]{Observa\c{c}\~{a}o}
\newenvironment{demonstracao}{\par
  \trivlist
  \item[\hskip\labelsep
        \itshape
    Dem.\textup{:}]\ignorespaces
}{\hfill$\halmos$\endtrivlist
}
\newcommand{\bibliografia}[1]{{\small \begin{thebibliography}{99} #1 \end{thebibliography}}}

\renewcommand\refname{Refer\^{e}ncias}
\renewcommand\figurename{Figura}
\renewcommand\tablename{Tabela}


\usepackage[portuguese]{babel}
\usepackage{ae}
\usepackage{graphicx}
\usepackage{epstopdf}
\usepackage{amssymb,amsmath}
\usepackage{bm} 
%\usepackage[utf8]{inputenc}  
\usepackage[latin1]{inputenc}
\usepackage{imakeidx} 
\makeindex[title={\'{I}ndice de Autores},columns=2] 

\usepackage{eurosym}
\usepackage{icomma}
\usepackage{url}
\usepackage{color}
\usepackage{subfigure}

%\usepackage{afterpage}
\usepackage{fancyhdr}
\pagestyle{fancy}
\pagestyle{empty}
%\usepackage[font=small,format=plain,labelfont=bf,up,up]{caption}
%\captionsetup[figure]{name=Figura}
\usepackage{amssymb}
\usepackage{wasysym}
\usepackage[usenames,dvipsnames]{xcolor}

%\usepackage{layout}
%\setlength{\oddsidemargin}{24pt}
%\setlength{\evensidemargin}{24pt}
%\setlength{\footskip}{50pt}
% \setlength{\textheight}{170pt}



% Nota: Os comandos de LateX para o trabalho est�o definindos no ficheiro Trabalho_MB.def


\setcounter {table}{0}
\setcounter{equation}{0}
\setcounter{figure}{0}
\setcounter{section}{0}

\pagestyle{plain}

\begin{document}

%inserir aqui o t�tulo do Trabalho
\titulo{T�tulo}


% inserir linhas com a forma abaixo por cada autor
% a identifica��o de cada autor deve ser separada por uma linha em branco
\autor{Autor 1 (nome)}{Autor 1 (apelido)}
\afil{afilia��o}{emailautor1@stats.univ.pt}

\autor{Autor 2 (nome)}{Autor 2 (apelido)}
\afil{afilia��o}{emailautor1@stats.univ.pt}
% mais um vez, uma linha em branco a preceder
% incluir uma ou duas palavras chave

\vspace{0.3cm}
\pchave{Estat�stica Bayesiana, Modelos Hier�rquicos, ...}

\medskip

\thispagestyle{empty}
\textbf{Resumo}: Um resumo em algumas linhas do trabalho proposto deve ser inserido aqui.

\newpage
\section{Introdu��o}
O resumo mais alargado do trabalho deve ser inserido aqui. A estrutura��o do texto utiliza a forma habitual em sec��es e, eventualmente, subsec��es.

\section{An�lise ...}

\subsection{subsec��o 1}
Desenvolvimento do trabalho efetuado, m�todos e modelos utilizados;\\

\subsection{subsec��o 2}
Dados, resultados.


\section{Principais conclus�es}

Uma pequena discuss�o final



\bibliografia{

\bibitem{artigo}
Leroux, BG., Lei, X., Breslow, N. \textit{Statistical Models in Epidemiology, the Environment, and Clinical Trials} chapter Estimation of Disease Rates in Small Areas: A new Mixed Model for Spatial Dependence, 179-191 . Springer-Verlag, New York, 2000.

\bibitem{livro}
 Blangiardo, M., Cameletti, M. \textit{Spatial and Spatio-temporal Bayesian Models with R-INLA}; Wiley, 2015.


}
\end{document} 